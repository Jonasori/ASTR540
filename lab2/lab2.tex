\documentclass[12pt]{article}
\usepackage[margin=1in]{geometry}
\usepackage{amsmath,amsthm,amssymb,amsfonts}
\usepackage{graphicx}

\newcommand{\N}{\mathbb{N}}
\newcommand{\Z}{\mathbb{Z}}

\newenvironment{part}[2][Part]{\begin{trivlist}
\item[\hskip \labelsep {\bfseries #1}\hskip \labelsep {\bfseries #2.}]}{\end{trivlist}}
%If you want to title your bold things something different just make another thing exactly like this but replace "problem" with the name of the thing you want, like theorem or lemma or whatever

\newenvironment{writeup}[2][Write-Up]{\begin{trivlist}
\item[\hskip \labelsep {\bfseries #1}\hskip \labelsep {\bfseries #2.}]}{\end{trivlist}}

\graphicspath{ {./} }


\begin{document}

%\renewcommand{\qedsymbol}{\filledbox}
%Good resources for looking up how to do stuff:
%Binary operators: http://www.access2science.com/latex/Binary.html
%General help: http://en.wikibooks.org/wiki/LaTeX/Mathematics
%Or just google stuff

\title{AST 540: Lab 2}
\author{Jonas Powell}
\maketitle

\begin{part}{Measuring the Visibility Function of a CFL}
\end{part}

\begin{writeup}{1}

\end{writeup}
\bigskip
\bigskip




\begin{part}{Measuring the Separation of Two CFLs Using the Interferometer}
\end{part}

\begin{writeup}{2}

\end{writeup}
\bigskip
\bigskip





\begin{part}{Measuring the VSRT Response to a Double Point Source of Variable Width}
\end{part}

\begin{writeup}{3}

\end{writeup}
\bigskip
\bigskip



\begin{part}{Measuring the Primary Beam of an LNBF}
\end{part}

\begin{writeup}{4}

\end{writeup}
\bigskip
\bigskip

\end{document}
